\chapter[Некоторые понятия линейной алгебры]{Некоторые понятия линейной алгебры\footnote{Неообходимо понимать, что, с одной строны, в данном разделе рассматриваются самые базовые понятия, которые даются практически в любом учебнике алгебры. С другой же стороны, определения данных понятий оказываются весьма вариативными и часто вызывают дискуссии. Напрмиер, интересный анализ тонкостей определения группы можно найти в первом и втором параграфах первого раздела книги Николая Александровича Вавилова ``Конкретная теория групп''~\cite{VavilovGroups}. Мы же дадим определения, удобные для дальнейшего изложения материала.}}\label{chpt:LinAlIntro}

При изложении ряда алгоритмов будут активно использоваться некоторые понятия и инструмены линейной алгебры, такие как моноид, полукольцо или матрица.
В данном разделе необходимые понятия будут определены и приведены некоторые примеры соответствующих конструкций. Для более глубокого изучения материала рекомендуются обратиться к соответствующим разделам алгебры.


\section{Бинарные операции и их свойства}


Введём понятие \textit{бинарной операции} и рассмотрим некоторые её свойства, такие как \textit{коммутативность} и \textit{ассоциативность}.

\begin{definition}[Двухместная функция] Функцию, принимающую два аргумента, $f: S \times K \to Q$ будем называть двухместной или функцией арности два.
Для запси таких функций будем использовать типичную нотацию: $c = f(a,b)$.
\end{definition}


\begin{definition}[Бинарная операция] 
Бинарная операция --- это двухместная функция, от которой дополнительно требуется, чтобы оба аргумента и результат лежали в одном и том же множестве: $f: S \times S \to S$. В таком случае говорят, что бинарная операция определена на некотором множестве $S$. Для обозначения произвольной бинарной операции будем использовать символ $\circ$ и пользоваться инфиксной нотацией для записи: $c = a \circ b$.
\end{definition}




\begin{definition}[Внешняя бинарная операция]
Внешняя бинарная операция --- это бинарная операция, у которой аргументы лежат в разных множествах, при этом результат --- в одном из этих множеств. Иными словами $\circ: K \times S \to S$, где $K$ может быть не равно $S$  --- внешняя бинарная операция.
\end{definition}


Необходимо помнить, что как функции, так и бинарные операции, могут быть частично определёнными (частичные функции, частичные бинарные операции). Типичным примером частично определённой бинарной операции является деление на целых числах: она не определена, если второй аргумент равен нулю.


Бинарные операции могут обладать некоторыми дополнительными свойствами, такими как \textit{коммутативность} или \textit{ассоциативность}, позволяющими преобразовывать выражения, составленные с использованием этих операций.


\begin{definition}[Коммутативность]
Бинарная операция $\circ : S \times S \to S$ называется коммутативной, если для любых  $x_1 \in S, x_2 \in S$ верно, что  $x_1 \circ x_2 = x_2 \circ x_1$.
\end{definition}

\begin{example} Рассмотрим несколько примеров коммутативных и некоммутативных операций.
	\begin{itemize}
		\item Опреация сложения на целых числах $+$ является коммутативной: известный ещё со школы перестановочный закон сложения.
		\item Операция конкатенации на строках $+$ не является коммутативной: $$``ab" + ``c" \ = ``abc" \neq ``c" + ``ab" \ = ``cab".$$
		\item Операция умножения на целых числах является коммутативной: известный ещё со школы перестановочный закон умножения.
		\item Операция умножения матриц (над целыми числами) $\cdot$ не является коммутативной:
		$$\begin{pmatrix} 
		1 & 1 \\ 0 & 0
		\end{pmatrix}
		\cdot
		\begin{pmatrix} 
		0 & 0 \\ 1 & 1
		\end{pmatrix}
		=
		\begin{pmatrix} 
		1 & 1 \\ 0 & 0
		\end{pmatrix}
		\neq
		\begin{pmatrix} 
		0 & 0 \\ 1 & 1
		\end{pmatrix}
		\cdot
		\begin{pmatrix} 
		1 & 1 \\ 0 & 0
		\end{pmatrix}
		=
		\begin{pmatrix} 
		0 & 0 \\ 1 & 1
		\end{pmatrix}
		.$$
	\end{itemize}
\end{example}

\begin{definition}[Ассоциативность]
Бинарная операция $\circ : S \times S \to S$ называется ассоциативной, если для любых  $x_1 \in S, x_2 \in S, x_3 \in S$ верно, что  $(x_1 \circ x_2) \circ x_3 = x_1 \circ (x_2 \circ x_3)$. Иными словами, для ассоциативной операции результат вычислений не зависит от порядка применения операций.
\end{definition}

\begin{example} Рассмотрим несколько примеров ассоциативных и неассоциативных операций.
	\begin{itemize}
		\item Опреация сложения на целых числах $+$ является ассоциативной.
		\item Операция умножения на целых числах является ассоциативной.
		\item Операция конкатенации на строках $+$ является ассоциативной: $$(``a" + ``b") + ``c" \ = ``a" + (``b" + ``c") = ``abc" .$$
		\item Операция возведения в степень (над целыми числами) $\hat{\mkern6mu}$ не является ассоциативной:
		$$(2\hat{\mkern6mu}2)\hat{\mkern6mu}3 = 4 \hat{\mkern6mu} 3 = 64 \neq 2\hat{\mkern6mu}(2\hat{\mkern6mu}3) = 2 \hat{\mkern6mu} 8  = 256.$$
	\end{itemize}
\end{example}


\begin{definition}[Дистрибутивность]
Говорят, что бинарная операция $\otimes : S \times S \to S$ является дистрибутивной относительно бинарной операции $\oplus : S \times S \to S$, если 
\begin{enumerate}
	\item Для любых $x_1,x_2,x_3 \in S, x_1 \otimes (x_2 \oplus x_3) = (x_1 \otimes x_2) \oplus (x_1 \otimes x_3)$ (дистрибутивность слева).
	\item Для любых $x_1,x_2,x_3 \in S, (x_2 \oplus x_3) \otimes x_1 = (x_2 \otimes x_1) \oplus (x_3 \otimes x_1)$ (дистрибутивнойть справа).
\end{enumerate}

Если операция $\otimes$ является коммутативной, то дистрибутивность слева и справа равносильны. 

\end{definition}

\begin{example} Рассмотрим несколько примеров дистрибутивных операций.

\begin{itemize}
	\item Умножение целых чисел дистрибутивно относительно сложения и вычитания: классический \textit{распределительный закон}, знакомый всем со школы.
	\item Операция деления (допустим, на действительных числах) не коммутативна. при этом, она дистрибутивна справа относительно сложения и вычитания, но не дистрибутивна слева.
	$$(a + b) / c = (a / c) + (b / c) $$
	но
	$$c / (a + b) \neq (c / a) + (c / b)\footnote{Здесь может быть уместно вспомнить правила сложения дробей. Дроби с общим знаминателем складывать проще как раз из-за дистрибутивности справа.}.$$
\end{itemize}

\end{example}

\begin{definition}[Идемпотентность]
Бинарная операция $\circ : S \times S \to S$ называется идемпотентной, если для любого  $x \in S$ верно, что  $x \circ x = x$.
\end{definition}



\begin{example} Рассмотрим несколько примеров идемпотентных операций.

\begin{itemize}
	\item Операция объединения множеств $\cup$ является идемпотентной: для любого множества $S$ верно, что $S \cup S = S$.
	\item Оперпция сложения на целых числах не является идемпотентной.
	\item Операция ``логическое и'' $\wedge$ является идемпотентной.
	\item Операция ``логическое или'' $\vee$ является идемпотентной.
\end{itemize}

\end{example}

\begin{definition}[Нейтральный элемент]
Пусть есть коммутативная бинарная операция $\circ$ на множестве $S$. Говорят, что $x\in S$ является нейтарльным элементом по операции $\circ$, если для любого $y\in S$ верно, что $x \circ y = y \circ x = y$. Если бинарная операция не является коммутативной, то можно пределить \textit{нейтральный слева} и \textit{нейтральный справа} элементы по аналогии.
\end{definition}


\section{Полугруппа}


\begin{definition}[Полугруппа]
Множество с заданной на нём ассоциативной бинарной операцией $(S,\cdot : S \times S \to S )$ называется полугруппой.
Если операция $\cdot$ является коммутативной, то говорят о \textit{коммутативной полугруппе}.
\end{definition}


\begin{example} Приведём несколько примеров полугрупп.
\begin{itemize}
	\item Положительные целые числа с операцией сложения являются полугруппой. Более того, коммутативной полугруппой.
	\item Целые числа с операцией взятия наибольшего из двух ($\max$) являются полугруппой. Более того, коммутативной полугруппой.
	\item Множество всех строк конечной длины (без пустой строки) над фиксированным алфавитом $\Sigma$ с операцией конкатенации является полугруппой. Так как конкатенация на строках не является коммутативной операцией, то и полугруппа не является коммутативной.	
\end{itemize}
\end{example}


\section{Моноид}


\begin{definition}[Моноид]
Моноидом называется полугруппа с нейтральным элементом. Если операция является коммутативной, то можно гворить о коммутативном моноиде.
\end{definition}

\begin{example} Приведём примеры моноидов, построенных на основе полугрупп из предыдущего раздела.

\begin{itemize}
	\item Неотрицательные целые числа с операцией сложения являются моноидом. Нейтральный элемент --- $0$.
	\item Целые числа, дополненные значением $-\infty$ (``минус-бесконесность'') с операцией взятия наибольшего из двух ($\max$) являются моноидом. Нейтральный элемент --- $-\infty$.
	\item Множество всех строк конечной длины с пустой строкой (строка длины 0) над фиксированным алфавитом $\Sigma$ и операцией конкатенации является моноидом. Нейтральный элемент --- пустая строка.
	\item Квадратные неотрицательные матрицы\footnote{Неотрицательной называется матрица, все элементы которой не меньше нуля.} фиксированного размера с операцией умножения задают моноид. Нейтральный элемент --- единичная матрица.
\end{itemize}
\end{example}


\section{Группа}

\begin{definition}[Группа]
Непустое\footnote{Требование непустоты здесь, как и далее, в определениях полукольца и кольца --- дискуссионный вопрос.} множество $G$ с заданной на нём бинарной операцией $\circ: {G} \times {G} \to {G}$ называется группой $(G ,\circ)$, если выполнены следующие аксиомы:
\begin{enumerate}
\item ассоциативность: $\forall (a,b,c\in G)\colon (a\circ b)\circ c = a\circ (b \circ c)$;
\item наличие нейтрального элемента: $ \exists e \in G \quad \forall a\in G\colon (e \circ a = a \circ e = a)$;
\item наличие обратного элемента: $ \forall a\in G\quad \exists a^{-1}\in G\colon (a \circ a^{-1}=a^{-1} \circ a = e)$.
\end{enumerate}

Иными словами, группа --- это моноид с дополнительным требованием наличия обратных элементов.

Если операция $\circ$ коммутативна, тоговорят, что группа \textit{Абелева}.
\end{definition}


\section{Полукольцо}

\begin{definition}[Полукольцо]

Непустое множество $R$ с двумя бинарными операциями $\oplus\colon R \times R \to R$ (часто называют умноженеим) и $\otimes \colon R \times R \to R$ (часто назывют сложением) называется полукольцом, если выполнены следующие условия.
\begin{enumerate}

\item $(R, \oplus)$ --- это коммутативный моноид, нейтральный элемент которого --- $\mathbb{0}$. Для любых $a,b,c \in R$:
\begin{itemize}
	\item $(a \oplus b) \oplus c = a \oplus (b \oplus c)$
	\item $\mathbb{0} \oplus a = a \oplus \mathbb{0} = a$
	\item $a \oplus b = b \oplus a$
\end{itemize}

\item $(R, \otimes)$ --- это моноид, нейтральный элемент которого --- $\mathbb{1}$. Для любых $a,b,c \in R$:
\begin{itemize}
	\item $(a \otimes b) \otimes c = a \otimes (b \otimes c)$
    \item $\mathbb{1} \otimes a = a \otimes \mathbb{1} = a$
\end{itemize}

\item $\otimes$ дистрибутивно слева и справа относительно $\oplus$:
\begin{itemize}
	\item $a \otimes (b \oplus c) = (a \otimes b) \oplus (a \otimes c)$
    \item $(a \oplus b) \otimes c = (a \otimes c) \oplus (b \otimes c)$
\end{itemize}


\item $\mathbb{0}$ является \textit{аннигилятором} по умножению: для любого $a \in R$ 
$\mathbb{0} \otimes a = a \otimes 0 = 0$

\end{enumerate}

\end{definition}

\begin{example}
Рассмотрим пример полукольца, а заодно покажем, что левая и правая дистрибутивность могут существовать независимо для некоммутативного умножения\footnote{Хороший пример того, почему левую и правую дистрибутивнойть в случае некоммутативного умножения нужно проверять независимо (правда, для колец), приведён Николаем Александровичем Вавмловым в книге ``Конкретная теория колец'' на странице 6~\cite{VavilovRings}.}.

В качестве $R$ возьмём множество множеств строк конечной длины над некоторым алфавитом $\Sigma$. В качестве сложения возьмём теоретико-множественное объединение: $\oplus  \equiv \cup$. Нейтральный элемент по сложению --- это пустое множество ($\varnothing$).
В качесве умножения возьмём конкатенацию множеств ($\otimes  \equiv \odot$) и оперделим её следующим образом:
$$ S_1 \odot S_2 = \{ w_1 \cdot w_2 \mid w_1 \in S_1, w_2 \in S_2\}$$, где $\cdot$ --- конкатенация строк. Нейтральным элементом по умножению будет являться множество из пустой строки: $\{\varepsilon\}$, где $\varepsilon$ --- обозначение для пустой строки.

Проверим, что $(R, \cup, \odot)$ действительно полукольцо по нашему определению.

\begin{enumerate}

\item $(R, \cup)$ --- действительно коммутативный моноид с нейтральным элементом $\varnothing$. Для любых $a,b,c \in R$ по свойствам теоретико-множественного объединения верно:
\begin{itemize}
	\item $(a \cup b) \cup c = a \cup (b \cup c)$
	\item $\varnothing \cup a = a \cup \varnothing = a$
	\item $a \cup b = b \cup a$.
\end{itemize}

\item $(R, \odot)$ --- вействительно моноид с нейтральным элементом $\{\varepsilon\}$. Для любых $a,b,c \in R$:
\begin{itemize}
	\item $(a \odot b) \odot c = a \odot (b \odot c)$ по определению $\odot$
    \item $\{\varepsilon\} \odot a = \{\varepsilon \cdot w \mid w \in a \} = \{w \mid w \in a \} = a \odot \{\varepsilon\} = a$
\end{itemize}
Вообще говоря, сконструированный нами моноид не является коммутативным: легко проверить, например, что для любых непустых $a,b \in R, a \neq b, a \neq \{\varepsilon\}, b \neq \{\varepsilon\}$: $a \cdot b \neq b \cdot a$ по причине некоммутативности конкатенации строк.

\item $\odot$ дистрибутивно слева и справа относительно $\cup$:
\begin{itemize}
	\item $a \odot (b \cup c) = \{ w_1 \cdot w_2 \mid  w_1 \in a, w_2 \in b \cup c\} = \{ w_1 \cdot w_2 \mid  w_1 \in a, w_2 \in b \} \cup  \{ w_1 \cdot w_2 \mid  w_1 \in a, w_2 \in c \} =  (a \odot b) \cup (a \odot c)$
    \item Аналогично, $(a \cup b) \odot c = (a \odot c) \cup (b \odot c)$
\end{itemize}
При этом, в общем случае, $a \odot (b \cup c) \neq (a \cup b) \odot c$.


\item $\varnothing$ является \textit{аннигилятором} по умножению: для любого $a \in R$ верно, что
$\varnothing \odot a =  \{ w_1 \cdot w_2 \mid w_1 \in \varnothing, w_2 \in a \} =  \{ w_1 \cdot w_2 \mid w_1 \in a, w_2 \in \varnothing \} = a \odot \varnothing = \varnothing$

\end{enumerate}

\end{example}

\section{Кольцо}


A ring is a set R equipped with two binary operations[a] + (addition) and $\otimes$ (multiplication) satisfying the following three sets of axioms, called the ring axioms[1][2][3]

R is an abelian group under addition, meaning that:
(a + b) + c = a + (b + c) for all a, b, c in R   (that is, + is associative).
a + b = b + a for all a, b in R   (that is, + is commutative).
There is an element 0 in R such that a + 0 = a for all a in R   (that is, 0 is the additive identity).
For each a in R there exists $-a$ in R such that $a + (-a) = 0$   (that is, $-a$ is the additive inverse of a).
R is a monoid under multiplication, meaning that:
(a $\otimes$ b) $\otimes$ c = a $\otimes$ (b $\otimes$ c) for all a, b, c in R   (that is, $\otimes$ is associative).
There is an element 1 in R such that a $\otimes$ 1 = a and 1 $\otimes$ a = a for all a in R   (that is, 1 is the multiplicative identity).[b]
Multiplication is distributive with respect to addition, meaning that:
a $\otimes$ (b + c) = (a $\otimes$ b) + (a $\otimes$ c) for all a, b, c in R   (left distributivity).
(b + c) $\otimes$ a = (b $\otimes$ a) + (c $\otimes$ a) for all a, b, c in R   (right distributivity).


\section{Поле}

\section{Матрицы и вектора}

Вектор

Матрица 

Про матричное произведение, тензорное произведение, ещё что-то.

%\section{Вопросы и задачи}
%\begin{enumerate}
%	\item Привидите примеры некоммутативных операций.
%	\item Привидите примеры ситуаций, когда наличие у бинарных операций каких-либо дополнитльных свойств (ассоциативности, коммутативности), позволяет строить более эффективные алгоритмы, чем в общем случае.
%\end{enumerate}