\chapter{Регулярные языки}


Регулярные языки --- иерархии, связанные с конечные автоматы, взаимные конвертации, основные свойства регулярных языков, такие как замкнутость относительно различных операций.


\section{Регулярные выражения}

\begin{definition}
    Регулярное множество (над алфавитом $\Sigma$) это:
    \begin{itemize}
        \item $\varnothing$
        \item $\{\varepsilon\}$
        \item $\{t\}$, $t \in \Sigma$
        \item $R_1 \cup R_2$, где $R_1$ и $R_2$ --- регулярные множества
        \item $R_1 \cdot R_2$, где $R_1$ и $R_2$ --- регулярные множества
        \item $R^*$, где $R$ --- регулярное множество
    \end{itemize}
\end{definition}


\begin{definition}
    Регулярное выражение.
    \begin{itemize}
        \item $\varepsilon$
        \item $t$
        \item $R_1 \mid R_2$
        \item $R_1 \cdot R_2$
        \item $R^*$
    \end{itemize}
\end{definition}

\section{Конечные автоматы}

\begin{definition}
    Конечный автомат.
\end{definition}

Пример КА.

Конфигурация, переход между конфигурациями.

Пример интерпретации конечного автомата.

Построение КА по регулярному выражению и регулярному выражению по КА. На производных.

Алгоритмы: проверка пустоты ... 

Примеры.



\section{Лево(право)линейные грамматики}

\begin{definition}
    Лево(право)линенйная грамматика. Правила вида  !!!
\end{definition}

Построение грамматики по автомату.

Пример построения грамматики по автомату.

Автомат по грамматике. 

\section{Лемма о накачке}

Лемма о накачке для регулярных языков.

\begin{lemma}
    Пусть $L$ --- регулярный язык над алфавитом $\Sigma$, тогда существует такое $n$, что для любого слова $\omega \in L$, $|\omega| \geq n$ найдутся слова $x,y,z\in \Sigma^*$, для которых верно: $xyz = \omega, y\neq \varepsilon,|xy|\leq n$ и для любого $k \geq 0$  $xy^kz \in L$.
\end{lemma}

Идея доказательства леммы о накачке.

\begin{enumerate}
    \item Так как язык регулярный, то для него можно построить автомат. В том числе, минимальный по количеству состояний.
    \item Возьмём в качестве $n$ количество состояний в автомате.
    \item Легко заметить, что для любой цепочки $w \in L, |w| > n$ путь в автомате, соответствующий принятию данной цепочки, будет содержать хотя бы один цикл.
          Действительно, в ориентированном графе с $n$ вершинами (а именно таким является автомат по построению) максимальная длина пути без повторных посещений вершин (соответственно, без циклов) не больше $n$.
    \item Выберем любой цикл. Он будет задавать искомые цепочки $x, y$ и $z$ так, как представлено на рисунке~\ref{!!!} 
\end{enumerate}


\section{Замкнутость регулярных языков относительно операций}

Доказательство замкнутости относительно операций. Алгоритмы для соответствующих операций.

Линейная алгебра для работы с регулярными языками: пересечение, замыкание.

Построение пересечения через тензорное произведение автоматов.

Пересечение через синхронный обход в ширину.

%\section{Вопросы и задачи}
%
%Построить базу.
%
%Научиться выполнять запросы через линейку. 