\chapter{Регулярные языки}



Регулярные языки, конечные автоматы, взаимные конвертации, замкнутость.

\begin{definition}
    Регулярное множество.
    \begin{itemize}
        \item $\varnothing$
        \item $\{\varepsilon\}$
        \item $\{t\}$
        \item $R_1 \cup R_2$
        \item $R_1 \cdot R_2$
        \item $R^*$
    \end{itemize}
\end{definition}

\section{Регулярные выражения}

\begin{definition}
    Регулярное выражение.
    \begin{itemize}
        \item $\varepsilon$
        \item $t$
        \item $R_1 \mid R_2$
        \item $R_1 \cdot R_2$
        \item $R^*$
    \end{itemize}
\end{definition}

\section{Конечные автоматы}

\begin{definition}
    Конечный автомат.
\end{definition}

Пример КА.

Конфигурация, переход между конфигурациями.

Пример интерпретации конечного автомата.

Построение КА по регулярке и регулярки по КА. На производных.

Алгоритмы: проверка пустоты ... 

Примеры.



\section{Лево(право)линейные грамматики}

\begin{definition}
    Лево(право)линенйная грамматика. Правила вида  !!!
\end{definition}

Построение грамматики по автомату.

Пример построения грамматики по автомату.

Автомат по грамматике. 

\section{Лемма о накачке}

Лемма о накачке для регулярных языков.

\begin{lemma}
    Лемма о накачке для регулярных языков.
\end{lemma}

Доказательство леммы о накачке для регулярных языков.

\section{Замкнутость регулярных языков относительно операций}

Доказательство замкнутости относительно операций. Алгоритмы для соответствующих операций.

Линейная алгебра для работы с регулярными языками: пересечение, замыкание.

Построение пересечения через тензорное произведение автоматов.

Пересечение через синхронный обход в ширину.

%\section{Вопросы и задачи}
%
%Построить базу.
%
%Научиться выполнять запросы через линейку. 