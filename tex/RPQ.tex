\chapter{Поиск путей с регулярными ограничениями}

\section{Достижимость между всеми парами вершин}

Через тензорное произведение.

Классическое построение пересечения автоматов строит их тензорное произведение.

Так как мы хотим отвечать ещё и на вопрос о достижимости, что нам надо ещё и транзитивное замыкание посчитать.

\section{Достижимость с несколькими источниками}

Достижимость от нескольких стартовых вершин через обход в ширину, основанный на линейной алгебре~\cite{9286186}.

Идея алгоритма основана на одновременном обходе в ширину графа и конечного автомата, построенного по грамматике.

В классической версии обхода в ширину, основанного на линейной алгебре, используется вектор, куда записывается фронт обхода графа.
Так, один раз перемножая этот вектор на матрицу смежности графа, можно совершать один шаг в обходе графа.
Покажем на примере, как данный метод может быть использован, когда мы накладываем дополнительные ограничения в виде регулярного языка на путь в графе. 

Для этого, во-первых, предъявим булевые представления для матриц смежности графа и автомата для регулярного языка. Затем, введем специальную блочно--диагональную матрицу 
для синхронизации обхода в ширину по двум матрицам смежности. Далее, попробуем наивно реализовать обход в ширину, и посмотрим, почему наивная реализация
может выдавать некорректный результат.
После этого перейдем к реализации обхода в ширину более продвинутым методом, который решает проблему наивного подхода.

\begin{example}
  Возьмём следующий граф.
    \begin{center}
      \label{input_rpq}
      \begin{tikzpicture}[node distance=2cm,shorten >=1pt,on grid,auto]
         \node[state] (q_0)   {$1$};
         \node[state] (q_1) [above=of q_0] {$2$};
         \node[state] (q_2) [right=of $(q_0)!0.5!(q_1)$] {$0$};
         \node[state] (q_3) [right=of q_2] {$3$};
          \path[->]
          (q_0) edge  node {b} (q_1)
          (q_1) edge  node[pos=0.3] {a} (q_2)
          (q_2) edge  node[pos=0.7] {a} (q_0)
          (q_2) edge[bend left]  node[above] {b} (q_3)
          (q_3) edge[bend left]  node {b} (q_2);
      \end{tikzpicture}
\end{center}

Его матрица смежности имеет следующий вид.
\[ G_1 =
\begin{pmatrix}
. & \{a\} & . & \{b\} \\
. & . & \{b\} & . \\
\{a\} & . & . & . \\
\{b\} & . & . & .
\end{pmatrix}
\]

Её булева декомпозиция по каждому символу выглядит следующим образом.
\begin{alignat*}{7}
& &&G_{0\_a} &&= \begin{pmatrix}
0 & 1       & 0 & 0       \\
0 & 0 & 0       & 0 \\
1  & 0 & 0 & 0       \\
0       & 0 & 0 & 0 \\
\end{pmatrix} \ \ \ \ &&G_{0\_b} &&= \begin{pmatrix}
0 & 0       & 0 & 1       \\
0       & 0 & 1       & 0 \\
0  & 0 & 0 & 0       \\
1       & 0 & 0 & 0 \\
\end{pmatrix}
\end{alignat*}

Зададим ограничения с помощью регулярного выражения $b^*ab$, которое представляется автоматом из трех последовательных состояний.

\begin{center}
  \begin{tikzpicture}[shorten >=1pt,on grid,auto]
  \node[state, initial]   (q_0) at (0,0)  {$0$};
  \node[state]             (q_1) at (2,0)  {$1$};
  \node[state, accepting] (q_2) at (4,0)  {$2$};
  \path[->]
  (q_0) edge  node {$a$} (q_1)
  (q_1) edge  node {$b$} (q_2);
  \draw (q_0) edge[loop above]  node {$b$} (q_0);
  \end{tikzpicture}
\end{center}


Автомат может быть задан матрицей смежности (с дополнительной информацией о стартовых и финальных состояниях).

Для регулярного выражения $b^*ab$ матрица смежности выглядит следующим образом (при этом нужно запомнить, что
состояние $0$ является начальным, $2$ --- конечным).

\[ G_2 =
\begin{pmatrix}
\{b\} & \{a\} & . \\
. & . & \{b\} \\
. & . & . 
\end{pmatrix}
\]

Нам будет необходима булева декомпозиция этой матрицы, и она выглядит следующим образом.

\begin{alignat*}{7}
  & &&R_{0\_a} &&= \begin{pmatrix}
    0 & 1 & 0 \\
    0 & 0 & 0 \\
    0 & 0 & 0 
    \end{pmatrix} \ \ \ \ &&R_{0\_b} &&= \begin{pmatrix}
      1 & 0 & 0 \\
      0 & 0 & 1 \\
      0 & 0 & 0 
      \end{pmatrix}
\end{alignat*}

Для синхронизации обхода составим набор блочно--диагональных матриц, каждая из которых --- это прямая сумма двух матриц:
$D_{0\_a} = R_{0\_a} \oplus G_{0\_a}$ и $D_{0\_a} = R_{0\_b} \oplus G_{0\_b}$.

\begin{alignat*}{7}
  & &&D_{0\_a} &&=
  \left(\begin{array}{c c c | c c c c}
    0 & 1 & 0 & 0 & 0 & 0 & 0 \\
    0 & 0 & 0 & 0 & 0 & 0 & 0 \\
    0 & 0 & 0 & 0 & 0 & 0 & 0 \\
    \hline
    0 & 0 & 0 & 0 & 1 & 0 & 0 \\
    0 & 0 & 0 & 0 & 0 & 0 & 0 \\
    0 & 0 & 0 & 1 & 0 & 0 & 0 \\
    0 & 0 & 0 & 0 & 0 & 0 & 0 
    \end{array}\right)
   \ \ \ \ &&D_{0\_b} &&=
   \left(\begin{array}{c c c | c c c c}
      1 & 0 & 0 & 0 & 0 & 0 & 0 \\
      0 & 0 & 1 & 0 & 0 & 0 & 0 \\
      0 & 0 & 0 & 0 & 0 & 0 & 0 \\
      \hline
      0 & 0 & 0 & 0 & 0 & 0 & 1 \\
      0 & 0 & 0 & 0 & 0 & 1 & 0 \\
      0 & 0 & 0 & 0 & 0 & 0 & 0 \\
      0 & 0 & 0 & 1 & 0 & 0 & 0 
      \end{array}\right)
\end{alignat*}

Пусть мы решаем частный случай задачи достижимости с несколькими стартовыми вершинами (multiple--source) 
--- достижимость с одной стартовой вершиной (single--source).

Пусть единственной начальной вершиной в графе будет вершина $0$.

Теперь создадим вектор $v = $ $\fbox{1 0 0} \fbox{1 0 0 0}$, где в первой части стоит единица на месте начального состояния $0$ в автомате. 
Во второй части содержится фронт обхода графа, на первом шаге это всегда множество стартовых вершин. В данном случае единица стоит на месте
единственной стартовой вершины --- $0$.

Совершим один шаг в обходе графа и получим новый фронт обхода графа.
\begin{alignat*}{7}
  a:\,\,
  & \begin{matrix}
    \fbox{1 0 0} \fbox{1 0 0 0}
    \end{matrix} &&
    \times
    \left(\begin{array}{c c c | c c c c}
    0 & 1 & 0 & 0 & 0 & 0 & 0 \\
    0 & 0 & 0 & 0 & 0 & 0 & 0 \\
    0 & 0 & 0 & 0 & 0 & 0 & 0 \\
    \hline
    0 & 0 & 0 & 0 & 1 & 0 & 0 \\
    0 & 0 & 0 & 0 & 0 & 0 & 0 \\
    0 & 0 & 0 & 1 & 0 & 0 & 0 \\
    0 & 0 & 0 & 0 & 0 & 0 & 0 
    \end{array}\right)
    &&= \begin{matrix}
        \fbox{0 1 0} \fbox{0 1 0 0} 
    \end{matrix}
\end{alignat*}

\begin{alignat*}{7}
  b:\,\,
  & \begin{matrix}
    \fbox{1 0 0} \fbox{1 0 0 0}
    \end{matrix}  &&
    \times
    \left(\begin{array}{c c c | c c c c}
      1 & 0 & 0 & 0 & 0 & 0 & 0 \\
      0 & 0 & 1 & 0 & 0 & 0 & 0 \\
      0 & 0 & 0 & 0 & 0 & 0 & 0 \\
      \hline
      0 & 0 & 0 & 0 & 0 & 0 & 1 \\
      0 & 0 & 0 & 0 & 0 & 1 & 0 \\
      0 & 0 & 0 & 0 & 0 & 0 & 0 \\
      0 & 0 & 0 & 1 & 0 & 0 & 0 
      \end{array}\right)
    &&= \begin{matrix}
        \fbox{1 0 0} \fbox{0 0 0 1}
    \end{matrix}
\end{alignat*}

Сложим два полученных вектора, чтобы получить новый фронт обхода графа: $\fbox{0 1 0} \fbox{0 1 0 0}$ + $\fbox{1 0 0} \fbox{0 0 0 1} = \fbox{1 1 0} \fbox{0 1 0 1}$.

То есть в наш фронт \fbox{0 1 0 1} попали вершины 1 и 3 соотвественно. А именно, мы совершили следующие переходы в графе и автомате.

\begin{alignat*}{7}
  \begin{tikzpicture}[node distance=2cm,shorten >=1pt,on grid,auto]
      \node[state, red] (q_0)   {$1$};
      \node[state] (q_1) [above=of q_0] {$2$};
      \node[state] (q_2) [right=of $(q_0)!0.5!(q_1)$] {$0$};
      \node[state, red] (q_3) [right=of q_2] {$3$};
      \path[->, red]
      (q_2) edge  node[pos=0.7] {a} (q_0)
      (q_2) edge[bend left]  node[above] {b} (q_3);
      \path[->]
      (q_0) edge  node {b} (q_1)
      (q_1) edge  node[pos=0.3] {a} (q_2)
      (q_3) edge[bend left]  node {b} (q_2);
  \end{tikzpicture}
  &\hspace{20px}
  \begin{tikzpicture}[shorten >=1pt,on grid,auto]
    \node[state, draw=none]      (q_3) at (0,0)  {$$}; % empty node for alignment
    \node[state, initial, red]   (q_0) at (0,1)  {$0$};
    \node[state, red]            (q_1) at (2,1)  {$1$};
    \node[state, accepting]      (q_2) at (4,1)  {$2$};
    \path[->, red]
    (q_0) edge  node {$a$} (q_1);
    \path[->]
    (q_1) edge  node {$b$} (q_2);
    \draw (q_0) edge[loop above, red]  node {$b$} (q_0);
  \end{tikzpicture}
\end{alignat*}
  
Совершим еще один шаг алгоритма. Теперь вектор $v$, на который мы умножаем матрицы, имеет следующий вид $\fbox{1 1 0} \fbox{0 1 0 1}$.

\begin{alignat*}{7}
  a:\,\,
  & \begin{matrix}
    \fbox{1 1 0} \fbox{0 1 0 1}
    \end{matrix}  &&
    \times
    \left(\begin{array}{c c c | c c c c}
    0 & 1 & 0 & 0 & 0 & 0 & 0 \\
    0 & 0 & 0 & 0 & 0 & 0 & 0 \\
    0 & 0 & 0 & 0 & 0 & 0 & 0 \\
    \hline
    0 & 0 & 0 & 0 & 1 & 0 & 0 \\
    0 & 0 & 0 & 0 & 0 & 0 & 0 \\
    0 & 0 & 0 & 1 & 0 & 0 & 0 \\
    0 & 0 & 0 & 0 & 0 & 0 & 0 
    \end{array}\right)
    &&= \begin{matrix}
        \fbox{0 1 0} \fbox{0 0 0 0} 
    \end{matrix}
\end{alignat*}

\begin{alignat*}{7}
  b:\,\,
  & \begin{matrix}
    \fbox{1 1 0} \fbox{0 1 0 1}
    \end{matrix}  &&
    \times
    \left(\begin{array}{c c c | c c c c}
      1 & 0 & 0 & 0 & 0 & 0 & 0 \\
      0 & 0 & 1 & 0 & 0 & 0 & 0 \\
      0 & 0 & 0 & 0 & 0 & 0 & 0 \\
      \hline
      0 & 0 & 0 & 0 & 0 & 0 & 1 \\
      0 & 0 & 0 & 0 & 0 & 1 & 0 \\
      0 & 0 & 0 & 0 & 0 & 0 & 0 \\
      0 & 0 & 0 & 1 & 0 & 0 & 0 
      \end{array}\right)
    &&= \begin{matrix}
        \fbox{1 0 1} \fbox{1 0 1 0}
    \end{matrix}
\end{alignat*}

$\fbox{0 1 0} \fbox{0 0 0 0}$  + $\fbox{1 0 1} \fbox{1 0 1 0} = \fbox{1 1 1} \fbox{1 0 1 0}$.
То есть в наш фронт \fbox{1 0 1 0} попали вершины 0 и 2 соотвественно. Мы совершили следующие переходы в графе и автомате.

\begin{alignat*}{7}
  \begin{tikzpicture}[node distance=2cm,shorten >=1pt,on grid,auto]
    \node[state, gray] (q_0)   {$1$};
    \node[state, teal] (q_1) [above=of q_0] {$2$};
    \node[state, teal] (q_2) [right=of $(q_0)!0.5!(q_1)$] {$0$};
    \node[state, gray] (q_3) [right=of q_2] {$3$};
      \path[->, gray]
      (q_2) edge  node[pos=0.7] {a} (q_0)
      (q_2) edge[bend left]  node[above] {b} (q_3);
      \path[->, red]
      (q_0) edge  node {b} (q_1)
      (q_3) edge[bend left]  node {b} (q_2);
      \path[->]
      (q_1) edge  node[pos=0.3] {a} (q_2);
  \end{tikzpicture}
  &\hspace{20px}
  \begin{tikzpicture}[shorten >=1pt,on grid,auto]
    \node[state, draw=none]            (q_3) at (0,0)  {$$}; % empty node for alignment
    \node[state, initial, red]         (q_0) at (0,1)  {$0$};
    \node[state, gray]                 (q_1) at (2,1)  {$1$};
    \node[state, accepting, teal]      (q_2) at (4,1)  {$2$};
    \path[->, gray]
    (q_0) edge  node {$a$} (q_1);
    \path[->, red]
    (q_1) edge  node {$b$} (q_2);
    \draw (q_0) edge[loop above, red]  node {$b$} (q_0);
  \end{tikzpicture}
\end{alignat*}

При этом, можно заметить, что мы достигли конечной вершины в автомате. Последний элемент левой части результирующего
вектора $\fbox{1 1 \textcolor{red}{1}} \fbox{1 0 1 0}$ отвечает за состояние 2, которое является конечным. А значит, обход необходимо остановить, и текущие вершины фронта
обхода графа записать в ответ.

Таким образом, вершины графа 0 и 2 являются ответом. Однако вершина 0 --- лишняя. Регулярное выражение $b^*ab$ не подразумевает, что
вершина 0 в графе может быть достигнута. Она могла бы быть достигнута по пустой строке в случае, если бы регулярное выражение имело вид $b^*$ или
по строке $aba$ в случае, если бы регулярное выражение имело вид $b^*aba$.

Это произошло, потому что в векторе $v$ должна кодироваться информация о паре --- вершине графа и состоянии автомата.
Достигнув вершины 0, мы оказались в конечном состоянии автомата, которое было получено с помощью другой вершины --- вершины 2.

Эту проблему можно решить, закодировав информацию о каждой такой паре в несколько векторов $v$, и ограничив левую
часть вектора $v$ таким образом, чтобы в ней всегда была лишь одна единица.

Тогда мы получим, что вектор $v$ вида $\fbox{1 0 0} \fbox{1 0 1 0}$ будет хранить информацию о парах (0, 0) и (0, 2),
где первый элемент пары --- состояние автомата, а второй --- вершина графа.

Аналогично, вектор $v$ вида $\fbox{0 1 0} \fbox{0 1 1 0}$ кодирует информацию о парах (1, 1) и (1, 2).
Вектор $v$ вида $\fbox{0 0 1} \fbox{0 0 1 1}$ кодирует информацию о парах (2, 2) и (2, 3).

Таким образом, мы будем понимать, в каком состоянии автомата мы находимся для каждой из вершин фронта обхода графа.

Рассмотрим, как это применяется в разработанном алгоритме, который представлен далее.

Предлагается ``расклеить'' $v$ в матрицу $M$, состоящую из трех векторов, добавив два вектора $\fbox{0 1 0} \fbox{0 0 0 0}$ и $\fbox{0 0 1} \fbox{0 0 0 0}$.
Во второй части этих векторов стоят нули, так как \fbox{0 1 0} и \fbox{0 0 1} кодируют состояния автомата 1 и 2, которые не являются начальными.

\begin{alignat*}{7}
  & &&M &&=\begin{matrix}
    \fbox{1 0 0} \fbox{1 0 0 0} \\ 
    \fbox{0 1 0} \fbox{0 0 0 0} \\
    \fbox{0 0 1} \fbox{0 0 0 0}
        \end{matrix}
\end{alignat*}

И совершать обход тем же самым образом, но сохранив с помощью матрицы $M$ дополнительную информацию о парах (состояние, вершина).

\begin{alignat*}{7}
  a:\,\,
  & \begin{matrix}
    \fbox{1 0 0} \fbox{1 0 0 0} \\ 
    \fbox{0 1 0} \fbox{0 0 0 0} \\
    \fbox{0 0 1} \fbox{0 0 0 0}
      \end{matrix}  &&
      \times
      \left(\begin{array}{c c c | c c c c}
      0 & 1 & 0 & 0 & 0 & 0 & 0 \\
      0 & 0 & 0 & 0 & 0 & 0 & 0 \\
      0 & 0 & 0 & 0 & 0 & 0 & 0 \\
      \hline
      0 & 0 & 0 & 0 & 1 & 0 & 0 \\
      0 & 0 & 0 & 0 & 0 & 0 & 0 \\
      0 & 0 & 0 & 1 & 0 & 0 & 0 \\
      0 & 0 & 0 & 0 & 0 & 0 & 0 
      \end{array}\right)
      &&= \begin{matrix}
        \fbox{0 1 0} \fbox{0 1 0 0} \\ 
        \fbox{0 0 0} \fbox{0 0 0 0} \\
        \fbox{0 0 0} \fbox{0 0 0 0}
    \end{matrix}
\end{alignat*}
  
\begin{alignat*}{7}
  b:\,\,
  & \begin{matrix}
    \fbox{1 0 0} \fbox{1 0 0 0} \\ 
    \fbox{0 1 0} \fbox{0 0 0 0} \\
    \fbox{0 0 1} \fbox{0 0 0 0}
      \end{matrix}  &&
      \times
      \left(\begin{array}{c c c | c c c c}
      1 & 0 & 0 & 0 & 0 & 0 & 0 \\
      0 & 0 & 1 & 0 & 0 & 0 & 0 \\
      0 & 0 & 0 & 0 & 0 & 0 & 0 \\
      \hline
      0 & 0 & 0 & 0 & 0 & 0 & 1 \\
      0 & 0 & 0 & 0 & 0 & 1 & 0 \\
      0 & 0 & 0 & 0 & 0 & 0 & 0 \\
      0 & 0 & 0 & 1 & 0 & 0 & 0 
      \end{array}\right)
      &&= \begin{matrix}
        \fbox{1 0 0} \fbox{0 0 0 1} \\ 
        \fbox{0 0 1} \fbox{0 0 0 0} \\
        \fbox{0 0 0} \fbox{0 0 0 0}
    \end{matrix}
\end{alignat*}

Для того, чтобы левая часть матрицы $M$ всегда оставалось единичной, нужно трансформировать в ней строчки особым образом.
Для этого нужно складывать только те вектора в правой части матрицы $M$, у которых в левой части единицы стоят на одинаковых позициях.
После чего переставлять строчки в $M$ так, чтобы левая часть матрицы $M$ принимала единичный вид.
Вектора с пустой левой частью нас при этом не интересуют.

Тогда правая часть матрицы $M$ будет кодировать текущий фронт обхода графа.

В нашем примере матрица $M$ для следующего шага обхода выглядит следующим образом.

\begin{alignat*}{7}
  & &&M &&=\begin{matrix}
    \fbox{1 0 0} \fbox{0 0 0 1} \\ 
    \fbox{0 1 0} \fbox{0 1 0 0} \\
    \fbox{0 0 1} \fbox{0 0 0 0}
        \end{matrix}
\end{alignat*}

Видно, что во фронт обхода графа попали вершины 1 и 3. В вершину 1 мы попали в состоянии 1, в вершину 3 --- в состоянии 0.

Совершаются следующие переходы в графе и автомате.

\begin{alignat*}{7}
  \begin{tikzpicture}[node distance=2cm,shorten >=1pt,on grid,auto]
      \node[state, red] (q_0)   {$1$};
      \node[state] (q_1) [above=of q_0] {$2$};
      \node[state] (q_2) [right=of $(q_0)!0.5!(q_1)$] {$0$};
      \node[state, red] (q_3) [right=of q_2] {$3$};
      \path[->, red]
      (q_2) edge  node[pos=0.7] {a} (q_0)
      (q_2) edge[bend left]  node[above] {b} (q_3);
      \path[->]
      (q_0) edge  node {b} (q_1)
      (q_1) edge  node[pos=0.3] {a} (q_2)
      (q_3) edge[bend left]  node {b} (q_2);
  \end{tikzpicture}
  &\hspace{20px}
  \begin{tikzpicture}[shorten >=1pt,on grid,auto]
    \node[state, draw=none]      (q_3) at (0,0)  {$$}; % empty node for alignment
    \node[state, initial, red]   (q_0) at (0,1)  {$0$};
    \node[state, red]            (q_1) at (2,1)  {$1$};
    \node[state, accepting]      (q_2) at (4,1)  {$2$};
    \path[->, red]
    (q_0) edge  node {$a$} (q_1);
    \path[->]
    (q_1) edge  node {$b$} (q_2);
    \draw (q_0) edge[loop above, red]  node {$b$} (q_0);
  \end{tikzpicture}
\end{alignat*}

Сделаем еще один шаг алгоритма и придем к конечному состоянию в автомате.

\begin{alignat*}{7}
  a:\,\,
  & \begin{matrix}
    \fbox{1 0 0} \fbox{0 0 0 1} \\ 
    \fbox{0 1 0} \fbox{0 1 0 0} \\
    \fbox{0 0 1} \fbox{0 0 0 0}
      \end{matrix}  &&
      \times
      \left(\begin{array}{c c c | c c c c}
      0 & 1 & 0 & 0 & 0 & 0 & 0 \\
      0 & 0 & 0 & 0 & 0 & 0 & 0 \\
      0 & 0 & 0 & 0 & 0 & 0 & 0 \\
      \hline
      0 & 0 & 0 & 0 & 1 & 0 & 0 \\
      0 & 0 & 0 & 0 & 0 & 0 & 0 \\
      0 & 0 & 0 & 1 & 0 & 0 & 0 \\
      0 & 0 & 0 & 0 & 0 & 0 & 0 
      \end{array}\right)
      &&= \begin{matrix}
        \fbox{0 1 0} \fbox{0 0 0 0} \\ 
        \fbox{0 0 0} \fbox{0 0 0 0} \\
        \fbox{0 0 0} \fbox{0 0 0 0}
    \end{matrix}
\end{alignat*}
  
\begin{alignat*}{7}
  b:\,\,
  & \begin{matrix}
    \fbox{1 0 0} \fbox{0 0 0 1} \\ 
    \fbox{0 1 0} \fbox{0 1 0 0} \\
    \fbox{0 0 1} \fbox{0 0 0 0}
      \end{matrix}  &&
      \times
      \left(\begin{array}{c c c | c c c c}
      1 & 0 & 0 & 0 & 0 & 0 & 0 \\
      0 & 0 & 1 & 0 & 0 & 0 & 0 \\
      0 & 0 & 0 & 0 & 0 & 0 & 0 \\
      \hline
      0 & 0 & 0 & 0 & 0 & 0 & 1 \\
      0 & 0 & 0 & 0 & 0 & 1 & 0 \\
      0 & 0 & 0 & 0 & 0 & 0 & 0 \\
      0 & 0 & 0 & 1 & 0 & 0 & 0 
      \end{array}\right)
      &&= \begin{matrix}
        \fbox{1 0 0} \fbox{1 0 0 0} \\ 
        \fbox{0 0 1} \fbox{0 0 1 0} \\
        \fbox{0 0 0} \fbox{0 0 0 0}
    \end{matrix}
\end{alignat*}

\begin{alignat*}{7}
  & &&M &&=\begin{matrix}
    \fbox{1 0 0} \fbox{1 0 0 0} \\ 
    \fbox{0 1 0} \fbox{0 0 0 0} \\
    \fbox{0 0 1} \fbox{0 0 1 0}
        \end{matrix}
\end{alignat*}

Видно, что мы достигли вершины 2 графа в конечном состоянии 2 автомата. При этом вершина 0 графа так же достигнута, как и в наивном варианте алгоритма, но
теперь известно, что это происходит в состоянии 0 автомата.

\begin{alignat*}{7}
  \begin{tikzpicture}[node distance=2cm,shorten >=1pt,on grid,auto]
    \node[state, gray] (q_0)   {$1$};
    \node[state, teal] (q_1) [above=of q_0] {$2$};
    \node[state, red] (q_2) [right=of $(q_0)!0.5!(q_1)$] {$0$};
    \node[state, gray] (q_3) [right=of q_2] {$3$};
      \path[->, gray]
      (q_2) edge  node[pos=0.7] {a} (q_0)
      (q_2) edge[bend left]  node[above] {b} (q_3);
      \path[->, red]
      (q_0) edge  node {b} (q_1)
      (q_3) edge[bend left]  node {b} (q_2);
      \path[->]
      (q_1) edge  node[pos=0.3] {a} (q_2);
  \end{tikzpicture}
  &\hspace{20px}
  \begin{tikzpicture}[shorten >=1pt,on grid,auto]
    \node[state, draw=none]            (q_3) at (0,0)  {$$}; % empty node for alignment
    \node[state, initial, red]         (q_0) at (0,1)  {$0$};
    \node[state, gray]                 (q_1) at (2,1)  {$1$};
    \node[state, accepting, teal]      (q_2) at (4,1)  {$2$};
    \path[->, gray]
    (q_0) edge  node {$a$} (q_1);
    \path[->, red]
    (q_1) edge  node {$b$} (q_2);
    \draw (q_0) edge[loop above, red]  node {$b$} (q_0);
  \end{tikzpicture}
\end{alignat*}

Таким образом, в ответ попадает вершина 2.
\end{example}

Перейдем к формальному описанию алгоритма.


Алгоритм принимает на вход граф $\mathcal{G}$, детерминированный конечный автомат $\mathcal{R}$, описывающий регулярную грамматику, и множество начальных вершин $V_{src}$ графа.

Граф $\mathcal{G}$ и автомат $\mathcal{R}$ можно представить в виде булевых матриц смежности. Так, в виде словаря для каждой метки графа заводится булева матрица смежности, на месте $(i, j)$ ячейки которой стоит 1, если $i$ и $j$ вершины графа соединены ребром данной метки. Такая же операция проводится для автомата грамматики $\mathcal{R}$.

Далее, мы оперируем с двумя словарями, где ключом является символ метки ребра графа или символ алфавита автомата, а значением --- соответствующая им булевая матрица.

Для каждого символа из пересечения этих множеств строится матрица $\mathfrak{D}$, как прямая сумма булевых матриц. То есть, строится матрица $\mathfrak{D} = Bool_{\mathcal{R}_a} \bigoplus Bool_{\mathcal{G}_a}$, которая определяется как

\begin{equation}
\mathfrak{D} = 
  \left[
    \begin{matrix}
        Bool_{\mathcal{R}_a} & 0\\
        0 & Bool_{\mathcal{G}_a}
    \end{matrix}
  \right]
\end{equation}

Где $\mathcal{R}_{a}$ и $\mathcal{G}_{a}$ матрицы смежности соответствующих символов автомата грамматики $\mathcal{R}$ и графа $\mathcal{G}$ для символа $a \in A_\mathcal{R} \cap A_\mathcal{G}$, $A_\mathcal{R} \cap A_\mathcal{G}$ --- пересечение алфавитов. Такая конструкция позволяет синхронизировать алгоритм обхода в ширину одновременно для графа и грамматики.

Далее вводится матрица $M$, хранящая информацию о фронте обхода графа. Она нужна для выделения множества пройденных вершин и не допускает зацикливание алгоритма.
\begin{equation}
M^{k \times (k + n)} =
  \left[
    \begin{matrix}
        Id_k & Matrix_{k \times n }
    \end{matrix}
  \right]
\end{equation}

Где $Id_k$ --- единичная матрица размера $k$, $k$ --- количество вершин в автомате $\mathcal{R}$, $Matrix_{k \times n }$ --- матрица, хранящая в себе маску пройденных вершин в автомате графа, $n$ --- количество вершин в графе $\mathcal{G}$.

\subsection{Выходные данные}

На выходе строится множество $\mathcal{P}$ пар вершин $(v, w)$ графа $\mathcal{G}$ таких, что вершина $w$ достижима из множества начальных вершин, при этом $v \in V_{src}$, $w \not\in V_{src}$. Это множество представляется в виде матрицы размера $|V|\times|V|$, где $(i,j)$ ячейка содержит 1, если пара вершин с индексами $(i, j) \in \mathcal{P}$.

\subsection{Процесс обхода графа}

Алгоритм обхода заключается в последовательном умножении матрицы $M$ текущего фронта на матрицу $\mathfrak{D}$. В результате чего, находится матрица $M'$ содержащая информацию о вершинах, достижимых на следующем шаге. Далее, с помощью операций перестановки и сложения векторов $M'$ преобразуется к виду матрицы $M$ и присваивается ей. Итерации продолжаются пока $M'$ содержит новые вершины, не содержащиеся в $M$. На листинге~\ref{BFSRPQ1} представлен этот алгоритм.

\begin{algorithm}[t]
  \caption{Алгоритм достижимости в графе с регулярными ограничениями на основе поиска в ширину, выраженный с помощью операций матричного умножения}\label{BFSRPQ1}
  \begin{algorithmic}[1]
    \Procedure{BFSBasedRPQ}{$\mathcal{R}=\langle Q, \Sigma, P, F, q \rangle,\mathcal{G}=\langle V, E, L \rangle, V_{src}$}
    \State $\mathcal{P}\gets~${Матрица смежности графа}
    \State $\mathfrak{D}\gets Bool_\mathcal{R} \bigoplus Bool_\mathcal{G}$\Comment{Построение матриц $\mathfrak{D}$}
    \State $M\gets CreateMasks(|Q|,|V|)$ \Comment{Построение матрицы $M$}
    \State $M'\gets SetStartVerts(M, V_{src})$  \Comment{Заполнение нач. вершин}
    
    \While{Матрица~$M$~меняется}{}
      \State $M\gets M'\langle\neg M\rangle$\Comment{Применение комплементарной маски}
      \ForAll{$a\in (\Sigma \cap L)$}
        \State $M'\gets M~$any.pair$~\mathfrak{D}$
        \Comment{Матр. умножение в полукольце}
        \State $M'\gets TransformRows(M')$\label{TransformRows}
        \Comment{Приведение $M'$ к виду $M$}
      \EndFor
        \State {$Matrix\gets extractRightSubMatrix(M')$}
        \State $V\gets Matrix.reduceVector()$ \Comment{Сложение по столбцам}
        \For{$k \in 0\dots|V_{src}|-1$}
            \State $W\gets\mathcal{P}.getRow(k)$
            \State $\mathcal{P}.setRow(k, V+W)$
      \EndFor
    \EndWhile
    \State \textbf{return} $\mathcal{P}$
    \EndProcedure
  \end{algorithmic}
\end{algorithm}

В алгоритме~\ref{BFSRPQ1}, в~\ref{TransformRows} строке происходит трансформация строчек в матрице $M'$. Это делается для того, чтобы представить полученную во время обхода матрицу $M'$, содержащую новый фронт, в виде матрицы $M$. Для этого требуется так переставить строчки $M'$, чтобы она содержала корректные по своему определению значения. То есть, имела единицы на главной диагонали, а все остальные значения в первых $k$ столбцах были нулями. Подробнее эта процедура описана в листинге~\ref{AlgoTransformRows}.

\begin{algorithm}[H]
  \caption{Алгоритм трансформации строчек}\label{AlgoTransformRows}
  \begin{algorithmic}[1]
    \Procedure{TransformRows}{$M$}
        \State{$T \gets extractLeftSubMatrix(M)$}
        \State{$Ix, Iy \gets$ итераторы по индексам ненулевых элементов $T$}
        \For{$i \in 0\dots|Iy|$}
            \State{$R\gets M.getRow(Ix[i])$}
            \State{$M'.setRow(Iy[i], R + M'.getRow(Iy[i]))$}
        \EndFor
    \EndProcedure
  \end{algorithmic}
\end{algorithm}

\pagebreak

\subsection{Модификации алгоритма}

Рассмотрим $V_{src}$ --- множество начальных вершин, состоящее из $r$ элементов. Для каждой начальной вершины $v_{src}^i \in V_{src}$ отметим соответствующие индексы в матрице $M$ единицами, получив матрицу $M(v_{src}^i)$,  и построим матрицу $\mathfrak{M}$ следующим образом.

\begin{equation}
\mathfrak{M}^{(k*r) \times (k + n)} =
  \left[
    \begin{matrix}
        M(v_{src}^1) \\
        M(v_{src}^2) \\ 
        M(\dots) \\
        M(v_{src}^r) \\
    \end{matrix}
  \right]
\end{equation}

Матрица $\mathfrak{M}$ собирается из множества матриц $M(v_{src}^i)$ и позволяет хранить информацию о том, из какой начальной вершины достигаются новые вершины во время обхода. 

\begin{algorithm}[t]
  \caption{Модификация алгоритма для поиска конкретной исходной вершины}\label{BFSRPQ2}
  \begin{algorithmic}[1]
    \Procedure{BFSBasedRPQ}{$\mathcal{R}=\langle Q, \Sigma, P, F, q \rangle,\mathcal{G}=\langle V, E, L \rangle, V_{src}$}
    \State $\mathcal{P}\gets~${Матрица смежности графа}
    \State $\mathfrak{D}\gets Bool_\mathcal{R} \bigoplus Bool_\mathcal{G}$
    \State $\mathfrak{M}\gets CreateMasks(|Q|,|V|)$
    \State $\mathfrak{M}'\gets SetStartVerts(\mathfrak{M}, V_{src})$  
    
    \While{Матрица~$\mathfrak{M}$~меняется}{}
      \State $\mathfrak{M}\gets \mathfrak{M}'\langle\neg\mathfrak{M}\rangle$
      \ForAll{$a\in (\Sigma \cap L)$}
        \State $\mathfrak{M}'\gets \mathfrak{M}~$any.pair$~\mathfrak{D}$
        \ForAll{$M \in \mathfrak{M}'$}
            \State $M\gets TransformRows(M)$
        \EndFor
      \EndFor
      \ForAll{$M_k \in \mathfrak{M}'$}
        \State $Matrix\gets extractSubMatrix(M)$
        \State $V\gets Matrix.reduceVector()$
        \State $W\gets\mathcal{P}.getRow(k)$
        \State $\mathcal{P}.setRow(k, V+W)$
      \EndFor
    \EndWhile
    \State \textbf{return} $\mathcal{P}$
    \EndProcedure
  \end{algorithmic}
\end{algorithm}

В листинге~\ref{BFSRPQ2} представлен модифицированный алгоритм. Основное его отличие заключается в том, что для каждой достижимой вершины находится конкретная исходная вершина, из которой начинался обход.

Таким образом, алгоритмы~\ref{BFSRPQ1}~и~\ref{BFSRPQ2} решают сформулированные в пункте \ref{sec:3.3} задачи достижимости.

