\chapter{Конъюнктивные и булевы грамматики}

Впервые конъюнктивные и булевы грамматики были предложены Александром Охотиным~\cite{DBLP:journals/jalc/Okhotin01,Okhotin:2003:BG:1758089.1758123}. Дадим определение конъюнктивной грамматики.

\begin{definition}
    \textit{Конъюнктивной грамматикой} называется $G = (\Sigma,N,P,S)$, где:
    \begin{itemize}
        \item $\Sigma$ и $N$ --- дизъюнктивные конечные непустые множества терминалов и нетерминалов.
        \item $P$ --- конечное множество продукций, каждая вида
        \[
        A\rightarrow \alpha_1\&...\&\alpha_n
        \]
        ,где $A \in N,n \geq 1$ и $\alpha_1,...,\alpha_n \in (\Sigma \cup N)^*$.
        \item $S \in N$  --- стартовый нетерминал.
    \end{itemize}
\end{definition}

Конъюнктивная грамматика генерирует строки, выводя их из начального символа, так же, как это происходит в контекстно-свободных грамматиках в параграфе~\ref{CFG}. Промежуточные строки, используемые в процессе вывода, являются формулами следующего вида:

\begin{definition}\label{Definition of conjunctive formula}
    Пусть $G = (\Sigma,N,P,S)$ --- конъюнктивная грамматика. Множество конъюнктивных формул $ \mathcal{F}$ определяется индуктивно:
    \begin{itemize}
        \item Пустая строка $\varepsilon$ --- конъюнктивная формула. 
        \item Любой символ из $(\Sigma \cup N)$ --- формула.
        \item Если $\mathcal{A}$ и $\mathcal{B}$ непустые формулы, тогда $\mathcal{AB}$ --- формула.
        \item Если $\mathcal{A}_1,\ldots,\mathcal{A}_n$ $(n \geqslant 1)$ --- формула, тогда $(\mathcal{A}_1\&\ldots\&\mathcal{A}_n)$ --- формула.
    \end{itemize}
\end{definition}

\begin{definition}
    Пусть $G = (\Sigma,N,P,S)$ --- конъюнктивная грамматика. Аналогично определению отношения непосредственной выводимости в контекстно-свободной грамматике~\ref{def derivability in CFG} определим $\xRightarrow[G]{}$ как отношение непосредственной выводимости на множестве конъюнктивных формул.
    \begin{itemize}
        \item Любой нетерминал в любой формуле может быть перезаписан телом любого правила для этого терминала заключенным в скобки. То есть для любых $s^{'},s^{''} \in (\Sigma \cup N \cup \{(, \&, )\})^*$ и $A\in N$, таких что $s^{'}As^{''}$ --- формула, и для всех правил вида $A \rightarrow \alpha_1\&\ldots\&\alpha_n \in P$, имеем $s^{'}As^{''}\xRightarrow[G]{}s^{'}(\alpha_1\&\ldots\&\alpha_n)s^{''}$. 
        \item Если формула содержит подформулу в виде конъюнкции одной или нескольких одинаковых терминальных строк, заключенных в скобки, тогда подформула может быть перезаписана терминальной строкой без скобок. То есть для любых $s^{'},s^{''} \in (\Sigma \cup N \cup \{(, \&, )\})^*$, $(n \geqslant 1)$ и $w \in \Sigma^*$, таких что $s^{'}(w\&\ldots\&w)s^{''}$ --- формула, имеем $s^{'}(w\&\ldots\&w)s^{''}\xRightarrow[G]{}s^{'}ws^{''}$.
    \end{itemize}
    Как и в случае контекстно-свободной грамматики обозначим $\xRightarrow[G]{}^*$ рефлексивное транзитивное замыкание отношения $\xRightarrow[G]{}$.
\end{definition}

\begin{definition}
    Пусть $G = (\Sigma,N,P,S)$ --- конъюнктивная грамматика. Язык, порождаемый формулой, это множество всех терминальных строк выводимых из этой формулы: $L_{G}(\mathcal{A}) = \{w\in\Sigma^* \mid \mathcal{A} \xRightarrow[G]{}^*w\}$. Очевидно, что язык порождаемый грамматикой, это язык порождаемый стартовым нетерминалом $S$ : $L(G) = L_{G}(S) = L(S)$.
\end{definition}

\begin{theorem}\label{Theorem language generated by a formula}
    Пусть $G = (\Sigma,N,P,S)$ --- конъюнктивная грамматика. Пусть $\mathcal{A}_1,\ldots,\mathcal{A}_n,\mathcal{B}$ --- формулы, $A \in N$, $a \in \Sigma$. Тогда,
    \begin{enumerate}
        \item $L(\varepsilon) = \{\varepsilon\}$.
        \item $L(a) = \{a\}$.
        \item $L(A) = \bigcup_{A \rightarrow \alpha_1\&\ldots\&\alpha_n \in P} L((\alpha_1\&\ldots\&\alpha_m))$.
        \item $L(\mathcal{AB}) = L(\mathcal{A})*L(\mathcal{B})$
        \item $L((\mathcal{A}_1\&\ldots\&\mathcal{A}_n)) = \bigcap_{i = 1}^{n}L(\mathcal{A}_i)$.
    \end{enumerate}
\end{theorem}

Теорема~\ref{Theorem language generated by a formula} уже подразумевает интерпретацию грамматики как системы уравнений. Используем математический подход, чтобы лучше охарактеризовать конъюнктивные языки с помощью систем уравнений.

\begin{definition}[Выражение]
    Пусть $\Sigma$ конечный непустой алфавит. Пусть $X = \{X_1,\ldots,X_N\}$ вектор переменных. Выражение над алфавитом $\Sigma$, зависящее от переменных $X$, определяется индуктивно:
    \begin{itemize}
       \item $\varepsilon$ --- выражение.
       \item Любой символ $a\in\Sigma$ --- выражение.
       \item Любая переменная $X_i\in X$ --- выражение.
       \item Если $\phi_1$ и $\phi_2$ выражения, то $\phi_1\phi_2, (\phi_1\mid\phi_2), (\phi_1\&\phi_2)$ также выражения.
    \end{itemize}
    Заметим, что любая формула, в терминах определения~\ref{Definition of conjunctive formula}, является выражением, где нетерминалы формулы это переменные выражения. С другой стороны, любое выражение, не содержащее дизъюнкции, формула.
\end{definition}

Предположим, что переменные $X_i$ приняли в качестве значений слова из языка над алфавитом $\Sigma$. Определим значение всего выражения.

\begin{definition}[Значение выражения]\label{Value of conjunctive expression}
    Пусть $L = (L_1,\ldots,L_n) (L_i \subseteq \Sigma^*)$ вектор из $n$ языков над $\Sigma$, где $n \geqslant 1$. Пусть $\phi$ выражение над $\Sigma$, зависящее от переменных $X_1,\ldots,X_n$. Значение выражения $\phi$ на векторе $L$ --- это язык над тем же алфавитом $\Sigma$. Обозначим его $\phi(L)$ и определим индуктивно на структуре выражения:
    \begin{itemize}
       \item $\varepsilon(L) = \{\varepsilon\}$.
       \item $a(L) = \{a\}$ для любого $a\in\Sigma$.
       \item $X_i(L) = L_i$ для любого $X_i \in X$.
       \item $\phi_1\phi_2 = \phi_1(L) * \phi_2(L), (\phi_1\mid\phi_2)(L) = \phi_1(L) \cup \phi_2(L), (\phi_1\&\phi_2)(L) = \phi_1(L) \cap \phi_2(L)$ для любых выражений $\phi_1$ и $\phi_2$.
    \end{itemize}
\end{definition}

Обобщим определение~\ref{Value of conjunctive expression} на случай вектора выражений.

\begin{definition}[Значение вектора выражений]
    Пусть $L = (L_1,\ldots,L_n) (L_i \subseteq \Sigma^*)$ вектор из $n$ языков над $\Sigma$, где $n \geqslant 1$. Пусть $\phi_1,\ldots,\phi_m$ выражения над $\Sigma$, зависящее от переменных $X_1,\ldots,X_n$. Значение вектора выражений $P = (\phi_1,\ldots,\phi_m)$ на векторе $L$ --- это вектор языков $P(L) = (\phi_1(L),\ldots,\phi_m(L))$ над тем же алфавитом $\Sigma$. 
\end{definition}

Зададим частичный порядок относительно включения $``\preccurlyeq"$ на множестве языков и расширим его на вектора языков длины $n$: $(L_1^{'},\ldots,L_n^{'})\preccurlyeq(L_1^{''},\ldots,L_n^{''})$ если и только если $L_i^{'} \subseteq L_i^{''}$ для любого $1\leqslant i \leqslant n$

\begin{definition}\label{Definition a conjuctive system of equations}
   $X = P(X)$ система уравнений над алфавитом $\Sigma$ и $X = \{X_1,\ldots,X_n\}$, где $P = (\phi_1,\ldots,\phi_n)$ вектор выражений над алфавитом $\Sigma$, зависящий от $X$.
   
   Вектор языков $L = (L_1,\ldots,L_n)$ является решением системы уравнений если $L = P(L)$.
   
   Наименьшее решение $L$ это вектор языков, такой что для любого другого сравнимого вектора языков $L^{'}$ выполняется $L \preccurlyeq L^{'}$.
\end{definition}

Заметим, что оператор $P$ на множестве $2^{\Sigma}\times\ldots\times2^{\Sigma}$, что решение системы~\ref{Definition a conjuctive system of equations} это неподвижная точка $P$ и что наименьшее решение системы это наименьшая неподвижная точка оператора $P$.

\begin{theorem}\label{Theorem of a least fixed point solution}
    Для любой системы из определения~\ref{Definition a conjuctive system of equations} с переменными $X_1,\ldots,X_n$, оператор $P = (\phi_1,\ldots,\phi_n)$ имеет наименьшую неподвижную точку $L = (L_1,\ldots,L_n) = \lim_{i\to\infty}P^{i}\underbrace{(\varnothing,\ldots,\varnothing)}_n$.
\end{theorem}

Приведем пример конъюнктивной грамматики.

\begin{example}[Пример конъюнктивной грамматики]
    Следующая конъюнктивная грамматика $G$ порождает язык $\{a^nb^nc^n\mid n \geq 0\}$:
    
    \begin{align*}
    1.\ S   &\to A B \& D C \\
    2.\ A  &\to a A \mid \varepsilon \\ 
    3.\ B &\to b B c \mid \varepsilon \\
    4.\ C   &\to c C \mid \varepsilon \\ 
    5.\ D   &\to aDb \mid \varepsilon
    \end{align*}
    
    Легко видеть, что $L(AB) = \{a^ib^jc^k\mid j = k\}$ и $L(DC) = \{a^ib^jc^k\mid i = j\}$. Тогда $L(S) = L(AB) \cap L(DC) = \{a^nb^nc^n\mid n \geq 0\}$. 
    
    В этой грамматике строка $abc$ может быть получена следующим образом. Для начала представим грамматику в виде системы уравнений:
    \begin{align*}
    S &= A B \cap D C \\ 
    A &= \{a\}A \cup \varepsilon \\ 
    B &= \{b\}B\{c\} \cup \varepsilon \\
    C &= \{c\}C \cup \varepsilon \\ 
    D &= \{a\}D\{b\} \cup \varepsilon
    \end{align*}
    Используя теорему~\ref{Theorem of a least fixed point solution}, будем итеративно вычислять $P^{i}\underbrace{(\varnothing,\ldots,\varnothing)}_5$. На каждом шаге будем подставлять все терминальные строки из языков, порожденных нетерминалами на предыдущем шаге, в соответствующие нетерминалы правой части каждого уравнения и записывать получившиеся терминальные строки в языки нетерминалов текущего шага. Продолжаем до тех пор пока язык, порождаемый нетерминалом $S$, не будет содержать терминальную строку $``abc''$.
    \begin{enumerate}
        \item На начальном этапе имеем $P^{0}(\varnothing,\ldots,\varnothing) = (S: \varnothing, A: \varnothing, B: \varnothing, C: \varnothing, D: \varnothing)$ 
        \item Подставляем в первое уравнение терминальные строки из шага 1 в соответствующие нетерминалы, т.е. 
        \begin{align*}
            S:  \varnothing &= \varnothing\varnothing \cap \varnothing\varnothing \\ 
            A: \{\varepsilon\} &= \{a\}\varnothing \cup \{\varepsilon\} \\ 
            B: \{\varepsilon\} &= \{b\}\varnothing\{c\} \cup \{\varepsilon\} \\
            C: \{\varepsilon\} &= \{c\}\varnothing \cup \{\varepsilon\} \\ 
            D: \{\varepsilon\} &= \{a\}\varnothing\{b\} \cup \{\varepsilon\}
        \end{align*}
        В конце итерации получаем $P^{1}(\varnothing,\ldots,\varnothing) = (S: \varnothing, A: \{\varepsilon\}, B: \{\varepsilon\}, C: \{\varepsilon\}, D: \{\varepsilon\})$
        \item Делаем еще одну итерацию,
        \begin{align*}
            S:  \{\varepsilon\} &= \{\varepsilon\}\{\varepsilon\} \cap \{\varepsilon\}\{\varepsilon\} \\ 
            A: \{a, \varepsilon\} &= \{a\}\{\varepsilon\} \cup \{\varepsilon\} \\ 
            B: \{bc, \varepsilon\} &= \{b\}\{\varepsilon\}\{c\} \cup \{\varepsilon\} \\
            C: \{c, \varepsilon\} &= \{c\}\{\varepsilon\} \cup \{\varepsilon\} \\ 
            D: \{ab, \varepsilon\} &= \{a\}\{\varepsilon\}\{b\} \cup \{\varepsilon\}
        \end{align*}
        В конце итерации получаем $P^{2}(\varnothing,\ldots,\varnothing) = (S: \{\varepsilon\}, A: \{a, \varepsilon\}, B: \{bc, \varepsilon\}, C: \{c, \varepsilon\}, D: \{ab, \varepsilon\})$
        \item Еще одна итерация,
        \begin{align*}
            S:  \{\fbox{abc}, \varepsilon\} &= \{a, \varepsilon\}\{bc, \varepsilon\} \cap \{ab, \varepsilon\}\{c, \varepsilon\} \\ 
            A: \{a, aa, \varepsilon\} &= \{a\}\{a, \varepsilon\} \cup \{\varepsilon\} \\ 
            B: \{bc, bbcc, \varepsilon\} &= \{b\}\{bc, \varepsilon\}\{c\} \cup \{\varepsilon\} \\
            C: \{c, cc, \varepsilon\} &= \{c\}\{c, \varepsilon\} \cup \{\varepsilon\} \\ 
            D: \{ab, aabb, \varepsilon\} &= \{a\}\{ab, \varepsilon\}\{b\} \cup \{\varepsilon\}
        \end{align*}
        В конце итерации получили $P^{3}(\varnothing,\ldots,\varnothing) = (S: \{\fbox{abc}, \varepsilon\}, A: \{a, aa, \varepsilon\}, B: \{bc, bbcc, \varepsilon\}, C: \{c, cc, \varepsilon\}, D: \{ab, aabb, \varepsilon\})$. Заметим, что терминальная строка $``abc"$ появилась в языке, который порождает стартовый нетерминал $S$. Т.е. терминальная строка $``abc"$ выводима в грамматике $G$, что и требовалось показать.
    \end{enumerate}
    
    Заметим, что строку $``abc"$ также можно получить применением правил вывода. здесь цифра над стрелкой соответствует номеру примененного правила. 
    \begin{align*}
        S &\xRightarrow{1}(AB\&DC) \\
        &\xRightarrow{2}(aAB\&DC) \xRightarrow{2} (a\varepsilon B\&DC) \\
        &\xRightarrow{3}(abBc\&DC) \xRightarrow{3}(ab\varepsilon c\&DC) \\
        &\xRightarrow{5}(abc\&aDbC) \xRightarrow{5}(abc\&a\varepsilon bC) \\
        &\xRightarrow{4}(abc\&abcC) \xRightarrow{4}(abc\&abc\varepsilon) \\
        &\Rightarrow(abc\&abc) \Rightarrow abc
    \end{align*}
\end{example}

\begin{example}
    Конъюнктивная грамматика $G$ для языка $L = \{wcw \mid w \in \{a, b\}^*\}$:
    \begin{align*}
    S &\to C \& D \\ 
    C &\to aCa \mid aCb \mid bCa \mid bCb \mid c \\ 
    D &\to aA\&aD \mid bB\&bD \mid cE \\
    A &\to aAa \mid aAb \mid bAa \mid bAb \mid cEa \\
    B &\to aBa \mid aBb \mid bBa \mid bBb \mid cEb \\
    E &\to aE \mid bE \mid \varepsilon
    \end{align*}
\end{example}

Подробнее о конъюнктивных грамматиках можно прочитать в статьях~\cite{DBLP:journals/jalc/Okhotin01, Okhotin2002, DBLP:journals/tcs/Okhotin03a, f60a33d409364914be560cac0e54b12c}.

Дадим определение булевой грамматики.

\begin{definition}
    \textit{Булевой грамматикой} называется $G = (\Sigma,N,P,S)$, где:
    \begin{itemize}
        \item $\Sigma$ и $N$ --- дизъюнктивные конечные непустые множества терминалов и нетерминалов.
        \item $P$ --- конечное множество продукций, каждая вида
        \[
        A\rightarrow \alpha_1\&...\&\alpha_m\&\neg\beta_1\&...\&\neg\beta_n
        \]
        ,где $A \in N, m, n >=0, m+n \geq 1$ и $\alpha_i,\beta_j \in (\Sigma \cup N)^*$.
        \item $S \in N$  --- стартовый нетерминал.
    \end{itemize}
\end{definition}

Приведем пример булевой грамматики.

\begin{example}
    Следующая булева грамматика порождает язык  $\{a^mb^nc^n\mid m,n \geq 0, m \neq n \}$:
    
    \begin{align*}
    S   &\to A B \& \neg D C \\ 
    A  &\to a A \mid \varepsilon \\ 
    B &\to b B c \mid \varepsilon \\
    C   &\to c C \mid \varepsilon \\ 
    D   &\to aDb \mid \varepsilon
    \end{align*}
    
    Очевидно, что $L(AB) = \{a^mb^nc^n\mid m,n \in \mathbb{N}\}$ и $L(DC) = \{a^nb^nc^m\mid m,n \in \mathbb{N}\}$. Тогда $L(AB)\cap\overline{L(DC)} = \{a^mb^nc^n\mid m,n \geq 0, m \neq n \}$.
\end{example}

Подробнее о булевых грамматиках можно прочитать в статьях~\cite{Okhotin:2003:BG:1758089.1758123,Okhotin:2014:PMM:2565359.2565379}.

Определим бинарную нормальную форму конъюнктивной грамматики.
\begin{definition}[Бинарная нормальная форма]
    Конъюнктивная грамматика $G = (\Sigma, N, P, S)$ находится в бинарной нормальной форме, если каждое правило из P имеет вид,
    \begin{itemize}
        \item $A \rightarrow B_1 C_1 \& \ldots\& B_m C_m$, где $m \geqslant 1; A,B_i,C_i \in N$.
        \item $A \rightarrow a$, где $A \in N, a \in \Sigma$.
        \item $S \rightarrow \varepsilon$, если только $S$ не содержится в правой части всех правил.
    \end{itemize}
\end{definition}

\begin{theorem}\label{Binary normal form conjunctive grammar theorem}
    Для каждой конъюнктивной грамматики $G$ можно построить конъюнктивную грамматику в бинарной нормальной форме $G^{'}$, такую что $L(G) = L(G^{'})$.
\end{theorem}
Доказательство теоремы~\ref{Binary normal form conjunctive grammar theorem} описано в статье~\cite{DBLP:journals/jalc/Okhotin01}.



%\section{Вопросы и задачи}
%\begin{enumerate}
%  \item !!! 
%  \item !!!
%\end{enumerate}
